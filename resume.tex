\documentclass[a4paper]{article}

\usepackage{cmap}
\usepackage[T1]{fontenc}
%\usepackage{helvet}
%\usepackage{tgadventor} % for the headings
%\usepackage{lmodern} % for the text
\usepackage{hyperref}
\usepackage{geometry}
\usepackage{sectsty}
\usepackage[usenames,dvipsnames]{color}
\usepackage[utf8]{inputenc}
\usepackage[english,russian]{babel}
%\usepackage{parcolumns}
\usepackage{multicol}
\usepackage{tabularx}
\usepackage{amssymb}

\usepackage{style}

%\allsectionsfont{\fontfamily{tgadventor}}

%\newenvironment{itemize*}%
%{\begin{itemize}%
%	\setlength{\itemsep}{1pt}%
%\setlength{\parskip}{0pt}%
%	\setlength{\itemindent}{0em}%
%\end{itemize}}
% Comment the following lines to use the default Computer Modern font
% instead of the Palatino font provided by the mathpazo package.
% Remove the 'osf' bit if you don't like the old style figures.
%\usepackage[T1]{fontenc}
%\usepackage[sc,osf]{mathpazo}
% Set your name here
%\def\name{\fontfamily{phv}\selectfont{Александр Соловец}}
\def\name{Александр Соловец}
\markright{\name}

% The following metadata will show up in the PDF properties
\hypersetup{
  colorlinks = true,
  urlcolor = dark-gray,
  pdfauthor = {\name},
  pdfkeywords = {math, mathematic, CS, software},
  pdftitle = {\name: Curriculum Vitae},
  pdfsubject = {Curriculum Vitae},
  pdfpagemode = UseNone
}

\begin{document}

\color{dark-gray}
\sf

\begin{center}
	{\Huge \bfseries \name}\\
	\vskip 1ex
  % \fontfamily{qag}\selectfont{\href{tel:+7-950-289-99-08}{+7 (950)
  % 289--99--08} {\small$\bullet$} Владивосток {\small$\bullet$}
  % \href{mailto:asolovets@gmail.com}{asolovets@gmail.com}}
  \href{tel:+7-950-289-99-08}{+7 (950)
  289--99--08} {\small$\bullet$} Владивосток {\small$\bullet$}
  \href{mailto:asolovets@gmail.com}{asolovets@gmail.com}
\end{center}

\small

	\section{Опыт}
		\begin{multicols}{2}
      \begin{project1}{Разработчик}{ФарПост, 2012}
				\begin{items}
          \item Провёл рефакторинг кода инструмента для статистического анализа
            и добавил показ доверительного интервала.
          \item Разработал сервис для подсчёта уникальных значений, общий объём
            которых не помещается воперативную память.
          \item Интегрировал ядро байесовского классификатора в веб-сервис,
            который классифицирует названия товаров.
				\end{items}
			\end{project1}

			\columnbreak

			\raggedcolumns
			\begin{project1}{Разработчик}{ВладЛинк, 2009 -- 2010}
				\begin{items}
          \item Оптимизировал алгоритм генерации тайловой карты, снизив время
            работы с недели до получаса.
          \item Разработал веб-приложение для отображения и редактирования
            схемы компьютерной сети городского интернет-провайдера с
            возможность в реальном времени видеть состояние обородувания и
            быстро диагностировать аварийный участок.
				\end{items}
			\end{project1}
    \end{multicols}

	\section{Проекты}
	\begin{multicols}{2}
		\raggedcolumns
		\begin{project2}{Review Board}{\begin{tabular}{@{}l|l}Google Summer of Code, 2011 & Python\\\end{tabular}}
			Интерфейс командной строки для системы проверки кода.
			\begin{items}
				\item Разработал и реализовал генератор кода для выполнения запросов на сервер.
				\item Добавил тесты в  унаследованный и новый код, увеличив покрытие с 31$\%$ до 42$\%$.
			\end{items}
		\end{project2}

		\begin{project2}{Freedroid RPG}{\begin{tabular}{@{}l|l}Google Summer of Code, 2010 & C\\\end{tabular}}
			Ролевая компьютерная игра.
			\begin{items}
				\item Реализовал алгоритм генерации подземелий в стиле <<NetHack>>.
				\item Разработал и реализовал алгоритм генерации открытых пространств, используя свойства самоподобия фракталов.
			\end{items}
		\end{project2}

		\columnbreak

		\begin{project2}{FriCAS}{\begin{tabular}{@{}l|l}Осенний марафон, 2010 & Axiom, Scheme\\\end{tabular}}
			Система комьютерной алгебры.
			\begin{items}
				\item Именьшил задержку в работе СКА на 70$\%$, заменив промежуточную процедуру конвертации \TeX{} нативной поддержкой синтаксиса редактора.
			\end{items}
		\end{project2}

		\begin{project2}{Компилятор}{\begin{tabular}{@{}l|l}Курсовая работа, 2008 & C\texttt{++}\\\end{tabular}}
			Транслятор ЯП <<Паскаль>> в язык ассемблера.
			\begin{items}
				\item Реализовал основные инструкции, включая процедуры и функции.
				\item Реализовал массивы, структуры и их суперпозиции.
				\item Реализовал передачу агрументов по ссылке.
			\end{items}
		\end{project2}
	\end{multicols}

	\section{Образование}
		%\begin{multicols}{2}
			%\raggedcolumns
			\begin{items}
				\item \textbf{Специалист, Прикладная математика и Информатика}\\
				Дальневосточный Федеральный университет, 2006 -- 2011.
			\end{items}
			%\columnbreak
			% this hack prevents above block from breaking onto two columns
			\hspace{10mm}
		%\end{multicols}

	\section{Навыки}
	\begin{multicols}{2}
		\raggedcolumns
		\begin{items}
      \item \textbf{Основы}: базовые знания алгоритмов и структур данных,
        понимание основ архитектуры x86, дискретная математика, ООП,
        функиональное и логическое программирование.

			\columnbreak

			\item \textbf{Языки}: C, C\texttt{++}, Java, Perl, Python, R.
      \item \textbf{Библиотеки}: Google Guava, Google Guice, Google Web
        Toolkit, Spring, Spring Batch, NumPy.
		\end{items}
	\end{multicols}

	\section{Достижения}
	\begin{items}
		\begin{multicols}{2}
			\raggedcolumns
			\item \textbf{Диплом \Rmnum{3} степени}\\ACM ICPC'2010 NEERC (Сибирская группа).
			\item \textbf{Диплом \Rmnum{3} степени}\\ACM ICPC'2009 NEERC (Сибирская группа).
			\columnbreak
			\item \textbf{Диплом \Rmnum{1} степени}\\<<Юный программист>>,\\Дальневосточный Государственный университет, 2006.
		\end{multicols}
	\end{items}

\end{document}
